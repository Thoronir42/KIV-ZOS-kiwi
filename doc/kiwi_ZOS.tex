\documentclass[12pt,a4paper]{article}
\usepackage[utf8]{inputenc}
\usepackage[czech]{babel}
\usepackage[T1]{fontenc}
\usepackage{amsmath}
\usepackage{amsfonts}
\usepackage{amssymb}
\usepackage{graphicx}
\usepackage{titlesec}
\usepackage[left=2cm,right=2cm,top=2cm,bottom=2cm]{geometry}
\usepackage{indentfirst}
\usepackage{listings}
\usepackage{color}
\usepackage{hyperref}

%Pravidlo pro řádkování
\renewcommand{\baselinestretch}{1.5}

%Pravidlo pro začínání kapitol na novém řádku
\let\oldsection\section
\renewcommand\section{\clearpage\oldsection}

%Formáty písem pro nadpisy (-změněno na bezpatkové \sffamily z původního \normalfont
\titleformat{\section}
{\sffamily\Large\bfseries}{\thesection}{1em}{}
\titleformat{\subsection}
{\sffamily\large\bfseries}{\thesubsection}{1em}{}
\titleformat{\subsubsection}
{\sffamily\normalsize\bfseries}{\thesubsubsection}{1em}{}

%Nastavení zvýrazňování kódu v \lslisting
\definecolor{mygreen}{rgb}{0,0.6,0}
\definecolor{mygray}{rgb}{0.5,0.5,0.5}
\lstset{commentstyle=\color{mygreen},keywordstyle=\color{blue},numberstyle=\tiny\color{mygray}}

\author{Jan Šmejkal}

\begin{document}

%-------------Úvodni strana---------------
\begin{titlepage}

\includegraphics[width=50mm]{img/FAV.jpg}
\\[160 pt]
\centerline{ \Huge \sc KIV/ZOS - Základy operačních systémů}
\centerline{ \huge \sc Semestrální práce }
\\[12 pt]
{\large \sc
\centerline{Zadání pro login kiwi}
}


{
\vfill 
\parindent=0cm
\textbf{Jméno:} Štěpán Ševčík\\
\textbf{Osobní číslo:} A13B0443P\\
\textbf{E-mail:} kiwi@students.zcu.cz\\
\textbf{Datum:} {\large \today\par} %datum
\textbf{www:} \url{https://logickehry.zcu.cz/}

}

\end{titlepage}

%------------------------------------------

%------------------Obsah-------------------
\newpage
\setcounter{page}{2}
\setcounter{tocdepth}{3}
\tableofcontents
%------------------------------------------

%--------------Text dokumentu--------------
\section{Zadání}
\subsection{1. část zadání}
Kontrola, zda každý řetěz FAT má správnou délku (odpovídá velikosti souboru v adresáři)
\subsection{2. část zadání}
Setřesení volného místa (volné místo na konci, ale bloky souborů nemusí jít za sebou).
\subsection{Další požadavky na aplikaci}
\begin{description}
\item Program vytvoří více vláken s možností zadat při/po spuštění počet paralelních entit
\item Zdrojový kód programu bude komentovaný - např. co dělá každá funkce
\item Bude ošetřeno špatné zadání vstupů
\item Součástí řešení bude skript v bashi, který umožní automatické spuštění všech testů nutných k vytvoření dokumentace (tj. např. spuštění s různým počtem vláken)
\item Dokumentace v elektronické podobě, popisující zejména použité konstrukce pro vytváření procesů/vláken/synchronizaci a použité algoritmy
\item
\end{description}

\section{Rozbor řešení úlohy}
\subsection{Spouštěč úlohy}
Aplikace obsahuje čtyři rutiny, mezi kterými lze vybírat požadovanou pomocí spouštěcích argumentů. Dvě rutiny jsou optimalizované verze zadaných rutin \textit{fat\_reader} a \textit{fat\_writer} a zbývající dvě rutiny vykonávají zadané úkony. Při spuštění se pomocí dedikované funkce určí požadovaná procedura a případně i počet vláken. Pokud je zadána neplatná kombinace parametrů nebo pokud je první parametr \textbf{help}, na obrazovku se vypíše instrukce pro použití a výkon programu skončí. Jinak se vykoná požadovaná instrukce definovaná v odpovídajícím souboru.
\subsection{1. část: Kontrola délky řetězce FAT}
Ověření správné délky řetězce fat lze provést dvěma různými způsoby: hrubě podle počtu clusterů, kde se uložená délka řetězce vydělí délkou clusteru a výsledek se porovná s počtem clusterů napočítaných v řetězu od začátku souboru do značky FAT\_FILE\_END, anebo jemně podle znaků v jednotlivých clusterech celého řetězce. Svoje řešení jsem vypracovával na jemném základu a celkovou délku souboru vypočítávám s přesností na znaky. Pokud vycházím z předpokladu že se clustery vyplňují od začátku a nenačíná se další cluster, pokud předchozí není zaplněný, stačí spočítat clustery, které nejsou v řetězci konečné, vynásobit jejich počet stanovenou délkou clusteru a připočíst k mezivýsledku délku posledního clusteru řetězce zjištěnou pomocí funkce \texttt{strlen}.
Bezpečnost paralelního výpočtu zajišťuje návrh ve stylu farmer worker, kdy se takzvaný farmer stará o přidělování souborů a jednotliví workeři provádějí výpočty. V tomto návrhovém vzoru nastávají dvě kritické sekce při čtení ze souboru: jedna je načítání \texttt{root\_directory} záznamů a druhá nastává při načítání clusterů řetězce. Obě tyto kritické sekce jsou řešeny mutexem, který hlídá přístup do souboru.
\subsection{2. část: Setřesení volného místa}
Aplikace funguje na dedikovaném modelu založeném na MVC architektuře. Při každém spuštění se načte odpovídající kontrolér, v němž se vykoná vhodná akce, která zpracuje uživatelův požadavek a připraví výsledek pro šablonovací část aplikace.

\section{Závěr}



%------------------------------------------

\end{document}